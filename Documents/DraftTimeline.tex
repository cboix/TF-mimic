% \documentclass[10pt]{article}            
% \usepackage[tmargin = 1.5cm, bmargin =1.5cm]{geometry}               
% \geometry{letterpaper}               
% \usepackage{amsmath}
% \usepackage{graphicx}                
% \usepackage{verbatim}
% \usepackage{amssymb}
% \usepackage{enumitem}
% \usepackage{tabularx}
% \usepackage{booktabs}
% \usepackage{subcaption}
% \usepackage{multirow}
% \usepackage{tikz}
% \usepackage[affil-it]{authblk}
% \usepackage{natbib}
% \setlist[enumerate]{itemsep=0mm}
\documentclass[landscape,a4paper]{article}
\usepackage{calendar} % Use the calendar.sty style
\usepackage[landscape,margin=0.5in]{geometry}
\usepackage{fancyhdr}
\setlength{\headheight}{15.2pt}
\pagestyle{fancy}
\fancyhead{} % delete current setting for header
\fancyhead[C]{\noindent Timeline for QCB 301 \hfill Carles Boix }
%\linespread{1.6}

\title{Timeline QCB 301}   
\author{Carles Boix}
\date{\today} %no date.

\begin{document}
%\pagestyle{empty} % Removes the page number from the bottom of the page

\noindent

\StartingDayNumber=1 % Calendar starting day, default of 1 means Sunday, 2 for Monday, etc
 
%----------------------------------------------------------------------------------------
%	TITLE SECTION
%----------------------------------------------------------------------------------------

\begin{center}
\textsc{\LARGE Rough Timeline for QCB 301}\\ % Title text
\end{center}
% \begin{minipage}[t]{.3\textwidth}
%  \begin{itemize}
%      \item[] \textbf{Plasmid construction}
%         \small
%     \item Obtain plasmids from Botstein Lab and McClean Lab.
%     \item Plan out and order primers.
%     \item If needed, order oligos for weaker binding DBDs.
%     \item PCR amplify fragments for \emph{Far1-22} and \emph{Far-WT} from Botstein plasmid.
%     \item PCR amplify fragments for \emph{DBDs} from McClean plasmid.
%     \item Homologous recombination of linearized plasmid w/ appropriate fragments (grow 2d). \emph{Do control recombination with Far WT to make sure that the plasmid itself is not toxic.}
%     \item Extract and transform bacteria, keeping a sample of yeast plasmid (grow 1d).
%     \item Extract plasmid (mini-prep) and send to sequencing (weekend).
% \end{itemize}
% \end{minipage}
% \normalsize %
% \hfill
% \begin{minipage}[t]{.3\textwidth}
% \begin{itemize}
%     \item[] \textbf{Blue Light Controls}
%         \small
%     \item Transform yeast with the two blue light plasmids.
%     \item Grow yeast (2d), expose to different BL. (see Liu,H. 2012).
%     \item Rates: 42 to 120 20 mmol m$^{-2}$ s$^{-1}$.
%     \item Sample at 15, 30, 45, 60 min (if needed, 2-6 hours)
%     \item Take OD$_{600}$ of each sample at time point.
%     \item Microarray of second set of samples after 1 and 2 hours.
% \end{itemize}
% \end{minipage}
% \normalsize %
% \hfill
% \begin{minipage}[t]{.3\textwidth}
% \begin{itemize}
%     \item[] \textbf{GAL Gradient}
%         \small
%     \item Transform yeast with plasmid if necessary.
%     \item Grow up yeast (2d).
%     \item Set up liquid minimal medium with galactose, 2\%.
%     \item Separate into 0 to 2\% glucose in 0.5\% intervals. 
%     \item Add to cells in mid-log phase and sample at time intervals and measure OD$_{600}$ over hours.
%     \item Also measure colony forming ability.
%     \item Visualize cells under microscope at several time points after adding to media. 
% \end{itemize}
% \end{minipage}
% \normalsize %
% \hfill
% \begin{minipage}[t]{.3\textwidth}
% \begin{itemize}
%     \item[] \textbf{Alternatives for Reducing Baseline}
%         \small
%     \item If needed (to do after GAL).
%     \item Using mut PCR to construct different FAR DBDs.
%     \item Construction of several different oligos for the FAR DBDs.
%     \item Weaker promoter to CRY2 and Zif construct.
%     \item Degradation of CRY2 and Zif construct.
%     \item Positive feedback of CRY2 and Zif construct.
% \end{itemize}
% \end{minipage}
% \hfill
% \begin{minipage}[t]{.3\textwidth}
% \begin{itemize}
%     \item[] \textbf{Testing Far1-22 constructs}
%         \small
%     \item For each of the DBD-Far1-22 constructs:
%     \item After sequence confirmation, grow to mid-log.
%     \item Place standard DBD under several (3 degree, 1 control, one room light) lights and measure growth OD$_{600}$ over several hours.
%     \item Visualize on slide, after 15 min and so on until \emph{each} of the samples 
%     \item Do this for the mutant constructs if sensitivity is an issue.
%     \item \emph{If} clearly stopping strains, see if we can cause them to grow again. 
%     \item If so, take \emph{RNA-seq} to characterize activity \textbf{before}, \textbf{after}, and \textbf{during} the arrest phase.
%     \item Construct alternative strains if any problems arise -- stopping = positive feedback; -- growing again = degradation (see plasmid construction and alternative methods).
% \end{itemize}
% \end{minipage}

%----------------------------------------------------------------------------------------
\begin{center}
\textsc{\Large Week of 10/14}\\ % Title text
\end{center}
\begin{calendar}{\hsize}
%   Sunday	
% By default all daily events are centered in the box, in order to bring them up use \vspace{2cm} after the event text; 
% You may need to change the 2cm
\day{}{} 

%	Monday
\day{}{
\textbf{Plasmids} \daysep Obtain GAL + Far1-22 from Botstein Lab\\
\textbf{Plasmids} \daysep Obtain construct from McClean Lab.\\[4pt]
\textbf{Prep} \daysep Plan out and order primers.\\
\textbf{Prep} \daysep If needed, order oligos for weaker binding DBDs.\\
} 

%	Tuesday
\day{}{
\textbf{Rap1} \daysep RNA purification and send to sequencing.\\[3pt]
} 
%	Wednesday
\day{}{ 
    \textbf{Controls} \daysep If I have them already, transform yeast with GAL + Far1-22 plasmid.\\ 
    Also with only the blue light construct plasmids.\\[4pt]
    \textbf{Shift Schedule} \daysep If this is done today, do the GAL experiments next Tuesday.
} 
%	Thursday
\day{}{
    \textbf{Today or next week.} \daysep PCR amplify fragments for \emph{Far1-22} and \emph{Far-WT} from Botstein plasmid.\\[3pt]
    \textbf{Today or next week.}    \daysep PCR amplify fragments for \emph{DBDs} from McClean plasmid.\\[3pt]
} 
%	Friday
\day{}{
    \textbf{By today} \daysep Have obtained plasmids\\ Planned recombination\\ Ordered primers/oligos.
} 
%	Saturday
\day{}{}
\finishCalendar
\end{calendar}

%------------------ %------------------ %------------------ %------------------ %------------------ %------------------ %------------------ 
\begin{center}
\textsc{\Large Week of 10/21}\\ % Title text
\end{center}
\begin{calendar}{\hsize}
%   Sunday	
\day{}{ } 
%	Monday
\day{}{
    \textbf{PCR} \daysep If not done yet, do PCR.\\ Also amplify Far1 WT\\
    \textbf{mutPCR} \daysep If I want to do this, do it today (recombination later or tomorrow).
    \textbf{Recombination} \daysep Put together Far1-22/WT + DBD strains.\\
} 
%	Tuesday
\day{}{
    \textbf{GAL gradient} \daysep Transform yeast with GAL plasmid (Botstein).\\
    \textbf{Light Control} \daysep Transform yeast with two blue light plasmids.\\
    \textbf{GAL media} \daysep Set up dosages of galactose and glucose for gradient work.\\
} 
%	Wednesday
\day{}{ 
    \textbf{Construction} \daysep Store some cells from recombination.\\ Extract plasmid and transform \emph{E. coli}.\\
    \textbf{Positive Feedback} \daysep Plan primers to recombine zinc finger promoter with ACT1 promoter in plasmid.\\
} 
%	Thursday
\day{}{
    \textbf{Mini-prep} \daysep Extract plasmid and purify \\ Send to sequencing.\\
    \textbf{GALp gradient} \daysep Store cultures for next week.\\
    \textbf{Light Control} \daysep Store cultures for next week.\\
} 
%	Friday
\day{}{
    \textbf{Mini-prep} \daysep Do today if a day behind.\\
    \textbf{By today} \daysep Recombined plasmid strains.\\ 
    Grow up yeast; transfer plasmid to bacteria.\\
    Send to sequencing?\\
} 
%	Saturday
\day{}{}
\finishCalendar
\end{calendar}

%------------------ %------------------ %------------------ %------------------ %------------------ %------------------ %------------------ 
\begin{center}
\textsc{\Large Week of 10/28}\\ % Title text
\end{center}
\begin{calendar}{\hsize}
%   Sunday	
% By default all daily events are centered in the box, in order to bring them up use \vspace{2cm} after the event text; 
% You may need to change the 2cm
\day{}{ 
    \textbf{Tests this week} \daysep If we have weak DBDs for Far1-22, also include in tests.
} 
%	Monday
\day{}{
    \textbf{GAL gradient} \daysep Taking transformed cells, set up several tubes of liquid culture to measure OD$_{600}$ over time.\\
    Come in earlier today ($\sim 12-1:00$).\\
    \textbf{GAL gradient} \daysep Meanwhile add portioned gal/glu to cells and measure speed of cell arrest by microscopy.\\
    \textbf{Protein Stability} \daysep Measure when 
} 
%	Tuesday
\day{}{
    \textbf{GAL} \daysep Yesterday's experiments should spill over to here.\\
    \textbf{Otherwise} \daysep Go over sequencing results and pick out cultures.\\
    \textbf{Tomorrow} \daysep Plan out the strains and the timing for tomorrow based on GAL results.\\
} 
%	Wednesday
\day{}{ 
    \textbf{Test Construct Arrest} \daysep Testing Far1-22, Far1WT, and (maybe) Far1-22 + weak DBDs.\\
    Grow at several light intensities and measure OD$_{600}$ over time intervals.\\
    Plate in parallel and leave over several hours to assess rate of colony formation at different lights.\\
    Visualize cells from liquid culture by microscopy at time points given by GAL work.
} 
%	Thursday
\day{}{
    \textbf{Test Construct Recovery} \daysep
    Using results from yesterday, choose light intensities that cause cell arrest, stimulate arrest, and then cover in foil for $\sim 120$ min to degrade Far1-22. \\ 
    Plate some cells and also visualize by microscopy. 
} 
%	Friday
\day{}{
    \textbf{Construct Recovery} \daysep Check plates; visualize cells on slide to check for synchronization.\\
    Check for residual effects by measuring OD$_{600}$ against a control and an uninduced sample.\\
    \textbf{By today} \daysep Have done at least GAL experiments and have construct arrest. Construct recovery would be optimal.\\
} 
%	Saturday
\day{}{}
\finishCalendar
\end{calendar}

\pagebreak
%------------------ %------------------ %------------------ %------------------ %------------------ %------------------ %------------------ 
\begin{center}
\textsc{\Large Week of 11/04}\\ % Title text
\end{center}
\begin{calendar}{\hsize}
%   Sunday	
% By default all daily events are centered in the box, in order to bring them up use \vspace{2cm} after the event text; 
% You may need to change the 2cm
\day{}{ 
    \textbf{Light Control} \daysep This week and next.\\[3pt]
    \textbf{Alternatives} \daysep Start this week.\\[8pt]
    Take any day to finish other work.
} 
%	Monday
\day{}
{
    \textbf{RNA-seq} \daysep RNA-seq at $0,\; 60,\; 120$ minutes of light intensity (IF a growth defect has been observed).
} 
%	Tuesday
\day{}{
    \textbf{Positive Feedback} \daysep Create fragment with zinc finger promoter (PCR) OR \emph{weak DBD}.\\
    Linearize plasmid extracted from \emph{E. coli} (sequencing step).\\
    \textbf{Prep} \daysep Blue light control cultures for tomorrow.
} 
%	Wednesday
\day{}{ 
    \textbf{Positive Feedback}\daysep Recombine to construct positive feedback CRY2 or CIB2 with DBD fragment and plasmid in yeast.\\
    Do so in the plasmid that does NOT contain Far1-22.\\
    \textbf{Blue Light} \daysep Taking control cells, measure growth relative to no light and no CRY2/CIB1 controls using $OD_{600}$.
} 
%	Thursday
\day{}{
    \textbf{Blue Light} \daysep Microarrays for controls and intensities (Not sure of procedure. Consult.)\\ Should also overlap into next week?
} 
%	Friday
\day{}{
    \textbf{Positive Feedback} \daysep Extract plasmid and transform \emph{E.coli} (or transform next week).\\
    Also transform yeast with the other half of the plasmid, containing Far1-22 (so that I can do experiments on toxicity of the cryptochromes).\\
    \textbf{By today} \daysep Prepared to test alternates. Half done with the blue light controls. Have finished the construct tests.\\
} 
%	Saturday
\day{}{}
\finishCalendar
\end{calendar}
%------------------ %------------------ %------------------ %------------------ %------------------ %------------------ %------------------
\begin{center}
\textsc{\Large Week of 11/11}\\ % Title text
\end{center}
\begin{calendar}{\hsize}
%   Sunday	
% By default all daily events are centered in the box, in order to bring them up use \vspace{2cm} after the event text; 
% You may need to change the 2cm
\day{}{ 
    \textbf{Light Control} \daysep Finish this week.\\[3pt]
    \textbf{Alternatives} \daysep Positive feedback AND weak DBDs.\\Also this week.\\[8pt]
    Take any day to finish other work.\\
} 
%	Monday
\day{}
{
    \textbf{Light Microarrays} \daysep Do today if not last Thursday.\\[3pt]
    \textbf{Positive Feedback} \daysep Extract plasmid from bacteria, miniprep, and send to sequence.\\
} 
%	Tuesday
\day{}{
    \textbf{Positive Feedback} \daysep Not waiting on sequencing, test growth by measuring OD$_{600}$, as with earlier constructs.\\
    Plate as well to see colony formation at different intensities.\\
} 
%	Wednesday
\day{}{ 
    \textbf{Alternatives and Recovery} \daysep Test recovery in similar way to construct in the alternative strategies.\\
    Plate cells and visualize by microscopy.\\
} 
%	Thursday
\day{}{
    \textbf{Alternatives and Recovery} \daysep Check plates. Check cultures. Visualize cells to check for synchronization.\\ Check residual effects as well by measuring growth.
} 
%	Friday
\day{}{
    \textbf{By today} \daysep Finished blue light testing.\\ 
    Tested alternate constructs.
} 
%	Saturday
\day{}{}
\finishCalendar
\end{calendar}
\begin{center}
    {\Large All weeks after this: analysis and buffer time.}\\
    {\large If able to, continue experiments with western blots and/or further characterization of blue light effect on yeast with different promoters to test for toxicity (such as the ones in alternative experiments).}
\end{center}

\end{document}
