\documentclass[12pt]{article}
\usepackage[lmargin =1in,rmargin=1in,tmargin =1in, bmargin =1in]{geometry}                
\geometry{letterpaper}               
\usepackage{graphicx}
\usepackage{subcaption}
\usepackage{multirow}
\usepackage{amssymb} \usepackage{fancyhdr}
\usepackage[affil-it]{authblk}
\usepackage{booktabs}
\usepackage{tabularx}
\usepackage{natbib}
\setlength{\headheight}{15.2pt}
\pagestyle{fancy}
\fancyhead{} % delete current setting for header
\fancyhead[C]{\noindent Cell Arrest in Yeast: An Optogenetic Aproach.\hfill Carles Boix }
%\linespread{1.6}

\begin {document}
% TITLE
% Background and significance.
% Hypothesis % GOAL?
% Aim 1, Aim 2, Aim 3.
% Broader Impacts
% References.

\textbf{Title:} Optogenetic-mediated cellular arrest and characterization of stress due to blue light and cryptochromes CRY2 and CIB1 in \emph{S. cerevisiae}. %Change blue light to BL + the cryptochromes

\textbf{Goal:} To create a blue light inducible construct to promote cell arrest in \emph{S. cervisiae}, quantitatively assess the response of yeast to blue light and the cryptochromes CRY2 and CIB1, and create a transcriptional switch to minimize baseline noise from the light-inducible transcription factor.

\textbf{Background and significance:} 
\emph{S. cerevisiae} cell cycle is well characterized and Far1, an inhibtor which interferes Cdc28, a cyclin which coordinates the cell cycle, has been shown to be an essential gene for cell cycle arrest (1). However, the current set of tools to induce cell cycle arrest and synchronize cells for study are limited, expensive, and non-gratuitous, in the best case inducing the full mating pathway and leading to cell deformity. Far1 mutant Far1-22, which lacks a phosphorylation site, has been shown to cause cell arrest in not only \textbf{a} and $\alpha$ cells but also in \textbf{a}$/\alpha$ diploids (2). I propose the use of an optogenetic transcription factor to quickly and efficiently induce cell cycle arrest by promoting Far1-22 in order to synchronize cells and study the cell cycle.

\emph{Arabidopsis} cryptochromes CRY2 and CIB1 (3), blue light inducible proteins, have been modified in \emph{S. cerevisiae} to create a light inducible transcription factor consisting of two subunits, with CRY2 paired with a zinc finger binding domain and CIB1 to an activating domain. This approach permits sensitive, fast, and nearly gratuitous transcriptional regulation of single genes in \emph{S. cerevisiae} by replacing their corresponding DNA binding domain with the zinc finger domain. In order to provide a more robust basis for this tool, I also propose to analyze the effects of blue light and the toxicity of the cryptochromes to \emph{S. cerevisiae} and address the issue of noisy baseline transcription of the construct.
%Recent optogenetics approaches:


%------------------------------------------------------------------------ 
\textbf{Aim 1:} \emph{Construction of a system to optogenetically promote cell arrest in yeast.}
I will construct a plasmid with a method to efficiently arrest \emph{S. cerevisiae} cells in the G1 phase by induction with blue light. A DNA binding domain corresponding to the binding motif of the CRY2/CIB1 complex will be placed in front of Far1-$22$ in a plasmid additionally containing the CRY2 subunit of the complex by homologous recombination in yeast (construct from Megan McClean). A plasmid with wild-type Far1p will be constructed to control for plasmid toxicity. After confirmation of plasmid sequence, transformed yeast will be exposed to a range of fluence rates of blue light (from $42$ to $120$ 20 mmol m$^{−2}$ s$^{−1}$) and cells will be visualized by microscopy to confirm cellular arrest. The sensitivity of growth to the level of Far1-22p will be assessed quantitatively by measuring OD$_{600}$ of cells grown at various light intensities and qualitatively by the rate of colony formation on plates with minimal media (in glycerol). As the blue light induced transcription factor may exhibit leakage, the same set of assessments will be performed on cells transformed with Far1-$22$ under the GAL promoter in minimal media with $2\%$ glucose and a range of glucose from $0\%$ to $2\%$ at $0.5\%$ intervals (plasmid from D. Botstein).
Far1-22p has a longer lifetime than Far1p ($\sim 120$ rather than $\sim 30$ minutes). Due to the stability of the protein and depending on the growth of the construct, I may have to reduce the baseline level of expression of the gene from the CRY2 promoter (as addressed in Aim 2). If due to stability of Far1-22p it is necessary to reduce the level of Far1-22p, it has been shown to be possible to degrade Far1-22p and not impact the cellular arrest phenotype by expressing non-localized cdc4, which degrades Far1p in the wild-type cell (4).

%------------------------------------------------------------------------ 
\textbf{Aim 2:} \emph{Creation of an optogenetic switch.}
I hypothesize that background light will induce a low level of transcription of any gene of interest from the CRY2 mediated promoter.  A system characterized by low baseline transcription and a rapid increase in promoter mediated transcription to quickly raise the expression of a gene of interest, the creation of an optogenetic switch, would provide a useful method for constitutively turning on genes in the cell. In particular, in the case of induction of genes which may interfere with cell growth, such as Far1-22, low baseline levels of transcription may interfere with normal function. To decrease the sensitivity of light induced transcription, I will place the DNA binding domain corresponding to the transcription factor construct in front of the CIB1 subunit to form a positive feedback mechanism to regulate the level of the transcription factor. Instead of constitutively expressing the TF, positive feedback coupled with transcriptional noise will create a bistable system where transcription will jump from low to high levels upon induction by light and will return to baseline after removing light stimulus.  I will assess expression levels with both a GFP reporter and by measuring cell arrest relative to controls with ACT1 promoters. If a suitable baseline level of transcription is not achieved, I will attempt to reduce the signal by expressing reporters from weaker DNA binding domains or prevent the accumulation of TF by rapidly degrading the CIB1 subunit by tagging it with a N-degron-tag and inducing the TEV protease (5).

%------------------------------------------------------------------------ 
\textbf{Aim 3:} \emph{Characterization of yeast response to CRY2 and CIB1 induction under blue light.}
To assess the impacts of using blue light to express genes with CRY2/CIB1 I will investigate the effect of blue light on yeast. Growth rate and gene expression will be measured at a range of fluence rates of blue light (from $42$ to $120$ 20 mmol m$^{−2}$ s$^{−1}$) in \emph{S. cerevisiae} cells transformed with the plasmids containing the constructs under the ACT1 promoter. I will measure growth using OD$_{600}$ readings and perform microarrays for the fluence rates with measured growth rate defects of more than $2\%$, using uninduced samples as control. To analyze microarray data, I will cluster genes by expression and search for affected pathways and potential binding sites to the CRY2/CIB1 construct. If growth defects are not correlated with gene expression, I will measure growth with CRY2 and CIB1 expressed under weaker constitutive promoters to test for toxicity. 

%\textbf{Conclusions:} 
\textbf{Broader Impacts:}
Completion of my aims will not only lead to the creation of a simple and efficient method for cell synchronization, but my analysis of the optogenetic transcription factor system will inform use of this tool in future single gene studies, in particular for network analysis. Furthermore, the creation of a bistable switch-like system for using the transcription factor will also add to the possibilities for network study; both in quickly activating and deactivating single gene.

\small
\textbf{References:}

(1) Chang, F., Herskowitz, I. (1992). Phosphorylation of FAR1 in response to alpha-factor: a possible requirement for cell-cycle arrest. Molecular biology of the cell, 3(4), 445.

(2) Henchoz, S., \ldots Peter, M. (1997). Phosphorylation-and ubiquitin-dependent degradation of the cyclin-dependent kinase inhibitor Far1p in budding yeast. Genes and development, 11(22), 3046-3060. %FAR1-22

(3) Liu, H.,\ldots Lin, C. (2008). Photoexcited CRY2 interacts with CIB1 to regulate transcription and floral initiation in Arabidopsis. Science, 322(5907), 1535-1539. % FIRST CIB / CRY2 paper.

(4) Blondel, M.,\ldots Peter, M. (2000). Nuclear-specific degradation of Far1 is controlled by the localization of the F-box protein Cdc4. The EMBO journal, 19(22), 6085-6097. % Can degrade.

(5) McIsaac,R.S.,\ldots Botstein,D. (2011) Fast-acting and nearly gratuitous induction of gene expression and protein depletion in Saccharomyces cerevisiae. Mol. Biol.  Cell., 22, 4447$-$4459 % TEV protease and N-degron tagging

\end{document}
