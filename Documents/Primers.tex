\documentclass[landscape,a4paper]{article}
\usepackage{calendar} % Use the calendar.sty style
\usepackage[landscape,margin=0.5in]{geometry}
\usepackage{fancyhdr}
\setlength{\headheight}{15.2pt}
\pagestyle{fancy}
\fancyhead{}
\fancyhead[C]{\noindent \today \hfill Carles Boix }

\date{\today} %no date.

\begin{document}
\noindent
\StartingDayNumber=1 % Calendar starting day, default of 1 means Sunday, 2 for Monday, etc
 
%	TITLE SECTION
\begin{center}
\textsc{\LARGE Primers for QCB 301}\\ % Title text
\end{center}

%----------------------------------------------------------------------------------------
\section*{Purposes:}
\begin{itemize}
    \item[\textbf{1.}] Primers for \emph{FAR1-wt} from genome to \emph{GAL4pr} plasmid.
    \item[\textbf{2.}] Primers for creating \emph{FAR1-22} mutation (to be used with \textbf{1.}).
    \item[\textbf{3.}] Primers for transferring \emph{FAR1-wt} and \emph{FAR1-22} from plasmid to genome.
    \item[\textbf{4.}] Primers for transferring \emph{ZEVpr} to genome.
    \item[\textbf{5.}] Primers for knocking out \emph{TRP1} with \emph{CORE}.
\end{itemize}
\section*{Primers:}
\begin{itemize}
    \item[\textbf{1.}] Forward pairs with \emph{HindIII} and backward pairs with \emph{BamHI}:

        \emph{These are the old primers. Sites can be found in FAR1 for each enzyme.}
        \[ 5' \qquad \emph{HindIII site } \cdots \emph{ Start of FAR1} \qquad 3' \] 
        \[ 5'\qquad \textbf{AATAAAGCTTATGAAGACACCAACAAGAGTTTC} \qquad 3' \] 
        \[ 5' \qquad \emph{BamHI site } \cdots \emph{ RC end of FAR1} \qquad 3' \] 
        \[ 5'\qquad \textbf{AATAGGATCCCTAGAGGTTGGGAACTTCC} \qquad 3' \] 

        \emph{These are the new primers. Only one site each in pMM86 and none in FAR}
        \[ 5' \qquad \emph{XhoI site } \cdots \emph{ Start of FAR1} \qquad 3' \] 
        \[ 5'\qquad \textbf{AATACTCGAGATGAAGACACCAACAAGAGTTTC} \qquad 3' \] 
        \[ 5' \qquad \emph{NotI site } \cdots \emph{ RC end of FAR1} \qquad 3' \] 
        \[ 5'\qquad \textbf{AATAGCGGCCGCCTAGAGGTTGGGAACTTCC} \qquad 3' \] 


        \emph{These are the new primers for recombination. The first matches the end of GAL. Note that it is best to cut with NotI or a site after XhoI but before NotI}

        \[ 5' \qquad \emph{End of pGAL } \cdots \emph{ Start of FAR1} \qquad 3' \] 
        \[ 5'\qquad \textbf{TATACTTTAACGTCAAGGAGAAAAAACTATACTCGAGATGAAGACACCAACAAGAGTTTC} \qquad 3' \] 
        \[ 5' \qquad \emph{RC After NotI site } \cdots \emph{ RC end of FAR1} \qquad 3' \] 
        \[ 5'\qquad \textbf{CGCGCAATTAACCCTCACTAAAGGGAACAAAAGCTGGAGCTCTAGAGGTTGGGAACTTCC} \qquad 3' \] 

 
    \item[\textbf{2.}] First will be used with second from \textbf{1.}, and second will be used with first from \textbf{2.}:
        \[ 5' \qquad \textbf{CAAATCTTGGCCTAATGATCCACCCACCAAGTTTGAAGAAAAC} \qquad 3' \]
        \[ 5' \qquad \textbf{GTTTTCTTCAAACTTGGTGGGTGGATCATTAGGCCAAGATTTG} \qquad 3' \] 

    \item[\textbf{3.}] Connect ends of \emph{FAR1} to ZEVpr (front) and pCORE (back)\\

           % R.C of the end of TRP1: 5'  AATAAATACTACTCAGTAATAAC 3'\\
            \[ 5' \qquad \emph{ZEVpr end } \cdots \emph{ Start of FAR1} \qquad 3' \] 
        \[ 5' \qquad \textbf{CGTCAAGGAGAAAAAACTATAGGTACCACTAGTATGGACGTATGAAGACACCAACAAGAGTTTC} \qquad 3' \] 
            \[ 5' \qquad \emph{RC of after Trp1 } \cdots \emph{ RC of end of FAR1} \qquad 3' \] 
        \[ 5' \qquad \textbf{GTGCACAAACAATACTTAAATAAATACTACTCAGTAATAACCTAGAGGTTGGGAACTTCC} \qquad 3' \] 

    \item[\textbf{4.}] Connect ZEVpr to \emph{FAR1} (back) and pCORE (back)\\

        % before TRP1 (or pCORE): 5' CACAAAGGCAGCTTGGAGT 3' -- has to attach to ZEVpr in front.\\
        % start of ZEV:  TTATATTGAATTTTCAAAAATTCTT ACTTTTTTTTTGGATGGACGCAAAGAA
        % end of ZEV: CGTCAAGGAGAAAAAACTATAGGTACCACTAGTATGGACGT
        % RC end of ZEV: ACGTCCATACTAGTGGTAC CTATAGTTTTTTCTCCTTGACG

\[ 5' \qquad \emph{before Trp1 } \cdots \emph{ Start of ZEVpr} \qquad 3' \] 
\[ 5' \qquad \textbf{GTGAGTATACGTGATTAAGCACACAAAGGCAGCTTGGAGTTTATATTGAATTTTCAAAAATTCTTA} \qquad 3' \] 
\[ 5' \qquad \emph{RC of start of FAR1 } \cdots \emph{ RC of end of ZEVpr} \qquad 3' \] 
\[ 5' \qquad \textbf{TGTATTTTTTTTTCAAACGAAACTCTTGTTGGTGTCTTCATACGTCCATACTAGTGGTAC} \qquad 3' \] 

    \item[\textbf{5.}] pCORE to TRP1 (Genomic)\\
%      On the URA3 side (start):
% GTGAGTATACGTGATTAAGCA CACAAAGGCAGCTTGGAGT   ATG $\Longrightarrow$ trp1

        \[ 5' \qquad \emph{before TRP1 } \cdots \emph{ Start of CORE} \qquad 3' \] 
        \[ 5' \qquad  \textbf{GTGAGTATACGTGATTAAGCACACAAAGGCAGCTTGGAGTTCCTTACCATTAAGTTGATC} \qquad 3' \] 

    % GENOMIC.
    %   On the kanMX4 side (end):\\
    %   end of trp1 --  TAG GTTATTACTGAGTAGTATTTATTTAAGT \\
    %   r.c. is:5' ACTTA AATAAATACTACTCAGTAATAAC CTA 3' $\Longrightarrow $ trp 1 complement\\

        \[ 5' \qquad \emph{RC of after TRP1 } \cdots \emph{ RC of end of CORE} \qquad 3' \] 
        \[ 5' \qquad  \textbf{GTGCACAAACAATACTTAAATAAATACTACTCAGTAATAACGAGCTCGTTTTCGACACTGG} \qquad 3' \] 

\end{itemize}

% Python to change:
% def transACTG(l):
%     if l == "A":
%         return("T")
%     if l == "C":
%         return("G")
%     if l == "T":
%         return("A")
%     if l == "G":
%         return("C")

% def conSEQ(a):
%     b = []
%     for i in range(0,len(a)):
%         b.append(    transACTG(a[i]))
%     b.reverse()
%     c = "".join(b)
%     print(c)

\end{document}
